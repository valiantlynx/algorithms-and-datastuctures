\documentclass{article}
\usepackage{amsmath}
\usepackage{graphicx}
\usepackage{listings}
\usepackage{hyperref}
\usepackage{geometry}
\geometry{a4paper, margin=1in}
\usepackage{xcolor}

\lstset{
    language=C++,
    basicstyle=\ttfamily\small,
    keywordstyle=\color{blue},
    stringstyle=\color{red},
    commentstyle=\color{green},
    breaklines=true,
    frame=single,
    showstringspaces=false,
}

\title{Implementing and Analyzing the Quick Sort with Random Pivot (Randomized Quick Sort) Algorithm in C++}
\author{Gormery K. Wanjiru}
\date{\today}

\begin{document}

\maketitle

\section{Introduction}
Randomized Quick Sort is a variation of the classic Quick Sort algorithm that selects a pivot randomly to improve average-case performance and mitigate the worst-case scenarios. This report describes the implementation of the Randomized Quick Sort algorithm in C++, analyzes its performance compared to Merge Sort and Normal Quick Sort, and provides a discussion of possible optimizations.

\section{Randomized Quick Sort Algorithm}
The Randomized Quick Sort algorithm works by selecting a pivot element randomly from the array, partitioning the array into elements less than and greater than the pivot, and recursively sorting the sub-arrays. This randomness helps to avoid the worst-case scenario that occurs when the pivot is chosen badly, which is common in already sorted or nearly sorted arrays.

The basic steps of Randomized Quick Sort are:
\begin{enumerate}
    \item Randomly select a pivot element from the array.
    \item Partition the array so that elements less than the pivot are on the left, and elements greater than the pivot are on the right.
    \item Recursively apply the same process to the left and right sub-arrays until the entire array is sorted.
\end{enumerate}

The C++ implementation of Randomized Quick Sort in this assignment can handle both integer and floating-point numbers, with appropriate error handling for invalid inputs.
The code is included in this report too.

\section{Implementation}
The implementation of Randomized Quick Sort in C++ is designed to handle arrays of both integers and floating-point numbers. It uses a random pivot to partition the array, and it employs recursion to sort the sub-arrays. Error handling is included to manage invalid inputs such as empty arrays or invalid data types.

Below is a "snippet" of the implemented code with comments explaining most of the logic and purpose of each component:

\begin{lstlisting}
#include <iostream>
#include <vector>
#include <cstdlib>
#include <ctime>
#include <algorithm>
#include <chrono>

using namespace std;

// Partition the array
// This function selects a random pivot, places it at the end, and partitions the array around the pivot
template <typename T>
int partition(vector<T> &arr, int low, int high) {
    int randomPivot = low + rand() % (high - low + 1);
    swap(arr[randomPivot], arr[high]); // Move the pivot(random) to the end
    T pivot = arr[high];
    int i = low - 1;

    // Partitioning the array around the pivot
    for (int j = low; j < high; ++j) {
        if (arr[j] <= pivot) {
            ++i;
            swap(arr[i], arr[j]);
        }
    }
    swap(arr[i + 1], arr[high]); // Place pivot in its correct position
    return i + 1;
}

// Randomized Quick Sort
// This function sorts the array by recursively partitioning it using a random pivot
template <typename T>
void randomizedQuickSort(vector<T> &arr, int low, int high) {
    if (low < high) {
        int pivotIndex = partition(arr, low, high); // Partition the array and get pivot index
        randomizedQuickSort(arr, low, pivotIndex - 1); // Recursively sort elements before pivot
        randomizedQuickSort(arr, pivotIndex + 1, high); // Recursively sort elements after pivot
    }
}

// Normal Quick Sort
// This function sorts the array using the classic Quick Sort algorithm
template <typename T>
int normalPartition(vector<T> &arr, int low, int high) {
    T pivot = arr[high];
    int i = low - 1;

    for (int j = low; j < high; ++j) {
        if (arr[j] <= pivot) {
            ++i;
            swap(arr[i], arr[j]);
        }
    }
    swap(arr[i + 1], arr[high]);
    return i + 1;
}

template <typename T>
void normalQuickSort(vector<T> &arr, int low, int high) {
    if (low < high) {
        int pivotIndex = normalPartition(arr, low, high);
        normalQuickSort(arr, low, pivotIndex - 1);
        normalQuickSort(arr, pivotIndex + 1, high);
    }
}

// Merge Sort
// This function merges two sorted halves of the array
template <typename T>
void merge(vector<T> &arr, int lb, int mid, int ub) {
    int i = lb, j = mid + 1, k = 0;
    vector<T> temp(ub - lb + 1);
    while (i <= mid && j <= ub) {
        if (arr[i] <= arr[j]) {
            temp[k++] = arr[i++];
        } else {
            temp[k++] = arr[j++];
        }
    }
    while (i <= mid) {
        temp[k++] = arr[i++];
    }
    while (j <= ub) {
        temp[k++] = arr[j++];
    }
    for (i = lb, k = 0; i <= ub; ++i, ++k) {
        arr[i] = temp[k];
    }
}

// This function recursively sorts the array using Merge Sort
template <typename T>
void mergeSort(vector<T> &arr, int lb, int ub) {
    if (lb < ub) {
        int mid = (lb + ub) / 2;
        mergeSort(arr, lb, mid); // Sort the first half
        mergeSort(arr, mid + 1, ub); // Sort the second half
        merge(arr, lb, mid, ub); // Merge the sorted halves
    }
}

// Function to handle user input and perform sorting
template <typename T>
void sortArray() {
    int n;
    cout << "Enter the number of elements in the array: ";
    cin >> n;
    if (n <= 0) {
        cout << "Invalid input. The array must contain at least one element." << endl;
        return;
    }

    vector<T> arr(n);
    cout << "Enter the elements of the array: " << endl;
    for (int i = 0; i < n; ++i) {
        cin >> arr[i];
    }

    vector<T> arrCopy1 = arr; // Create a copy of the array for Merge Sort comparison
    vector<T> arrCopy2 = arr; // Create a copy of the array for Normal Quick Sort comparison

    // Measure Randomized Quick Sort time
    auto start = chrono::high_resolution_clock::now();
    randomizedQuickSort(arr, 0, n - 1);
    auto end = chrono::high_resolution_clock::now();
    auto quickSortDuration = chrono::duration_cast<chrono::microseconds>(end - start).count();

    // Output the sorted array using Randomized Quick Sort
    cout << "The sorted array using Randomized Quick Sort is: " << endl;
    for (const auto &element : arr) {
        cout << element << " ";
    }
    cout << endl;
    cout << "Time taken by Randomized Quick Sort: " << quickSortDuration << " microseconds" << endl;

    // Measure Merge Sort time
    start = chrono::high_resolution_clock::now();
    mergeSort(arrCopy1, 0, n - 1);
    end = chrono::high_resolution_clock::now();
    auto mergeSortDuration = chrono::duration_cast<chrono::microseconds>(end - start).count();

    // Output the sorted array using Merge Sort
    cout << "The sorted array using Merge Sort is: " << endl;
    for (const auto &element : arrCopy1) {
        cout << element << " ";
    }
    cout << endl;
    cout << "Time taken by Merge Sort: " << mergeSortDuration << " microseconds" << endl;

    // Measure Normal Quick Sort time
    start = chrono::high_resolution_clock::now();
    normalQuickSort(arrCopy2, 0, n - 1);
    end = chrono::high_resolution_clock::now();
    auto normalQuickSortDuration = chrono::duration_cast<chrono::microseconds>(end - start).count();

    // Output the sorted array using Normal Quick Sort
    cout << "The sorted array using Normal Quick Sort is: " << endl;
    for (const auto &element : arrCopy2) {
        cout << element << " ";
    }
    cout << endl;
    cout << "Time taken by Normal Quick Sort: " << normalQuickSortDuration << " microseconds" << endl;
}

int main() {
    srand(time(0)); // Seed the random number generator

    char type;
    cout << "Enter 'i' for integers or 'f' for floating-point numbers: ";
    cin >> type;

    if (type == 'i') {
        sortArray<int>(); // Sort an array of integers
    } else if (type == 'f') {
        sortArray<float>(); // Sort an array of floating-point numbers
    } else {
        cout << "Invalid input type." << endl;
    }

    return 0;
}
\end{lstlisting}

\section{Complexity Analysis}
The average-case time complexity of Randomized Quick Sort is $O(n \log n)$, which is the same as the classic Quick Sort. However, by selecting a random pivot, the probability of encountering the worst-case time complexity of $O(n^2)$ is significantly reduced.
But let's go deeper...
\subsection{Mathematical Proof of Complexity}
The recurrence relation for Randomized Quick Sort is given by:
\begin{equation}
    T(n) = T(k) + T(n - k - 1) + O(n)
\end{equation}
where $k$ is the position of the pivot after partitioning. The expectation of $T(n)$ can be analyzed by averaging over all possible choices of the pivot. Let us note the expected running time as $E[T(n)]$:
\begin{equation}
    E[T(n)] = \frac{1}{n} \sum_{k=0}^{n-1} \left( E[T(k)] + E[T(n - k - 1)] \right) + O(n)
\end{equation}
Simplifying this recurrence yields an expected time complexity of $O(n \log n)$. This is significantly better than the worst-case complexity of the normal Quick Sort, which occurs when the pivot is always the smallest or largest element, resulting in $O(n^2)$.

\section{Performance Analysis}
The performance of Randomized Quick Sort was compared with Merge Sort and Normal Quick Sort using execution time as the metric. The experiments were conducted on a few  input sizes, and the results are summarized in Table \ref{tab:performance}.

\begin{table}[h]
    \centering
    \begin{tabular}{|c|c|c|c|}
        \hline
        Algorithm & Average Time (Microseconds) & Best Case & Worst Case \\
        \hline
        Randomized Quick Sort & 1-3 & $O(n \log n)$ & $O(n^2)$ \\
        Merge Sort & 4-9 & $O(n \log n)$ & $O(n \log n)$ \\
        Normal Quick Sort & 0-1 & $O(n \log n)$ & $O(n^2)$ \\
        \hline
    \end{tabular}
    \caption{Performance Comparison of Sorting Algorithms}
    \label{tab:performance}
\end{table}

\section{Example Outputs}
Below are the sample outputs observed during testing:

\subsection{Floating-Point Numbers}
\begin{verbatim}
Enter 'i' for integers or 'f' for floating-point numbers: f
Enter the number of elements in the array: 6
Enter the elements of the array:
000.00
0400.00
4
44
45.6
0.00010
The sorted array using Randomized Quick Sort is:
0 0.0001 4 44 45.6 400
Time taken by Randomized Quick Sort: 1 microseconds
The sorted array using Merge Sort is:
0 0.0001 4 44 45.6 400
Time taken by Merge Sort: 9 microseconds
The sorted array using Normal Quick Sort is:
0 0.0001 4 44 45.6 400
Time taken by Normal Quick Sort: 1 microseconds
\end{verbatim}

\subsection{Integer Numbers}
\begin{verbatim}
Enter 'i' for integers or 'f' for floating-point numbers: i
Enter the number of elements in the array: 6
Enter the elements of the array:
45
44
4
2
0400
000
The sorted array using Randomized Quick Sort is:
0 2 4 44 45 400
Time taken by Randomized Quick Sort: 3 microseconds
The sorted array using Merge Sort is:
0 2 4 44 45 400
Time taken by Merge Sort: 4 microseconds
The sorted array using Normal Quick Sort is:
0 2 4 44 45 400
Time taken by Normal Quick Sort: 0 microseconds
\end{verbatim}

\section{Discussion and Optimization}
Randomized Quick Sort can be further optimized by employing hybrid approaches, such as switching to Insertion Sort for small sub-arrays. Additionally, choosing the median of three random elements as the pivot can further enhance performance by improving pivot quality.
I would also like to automate the test to be able to run hundreds of thousands of test( arrays of input arrays sizes) and average them.

\section{Conclusion}
Randomized Quick Sort provides an efficient and simple way to sort arrays, particularly when the input data is randomly distributed - the elements in the input array are in no specific order. Compared to Merge Sort and Normal Quick Sort, it "generally" offers better performance due to its ability to avoid the worst-case time complexity with high probability. However, Merge Sort remains advantageous for guaranteed time complexity and stability currently.

\section*{References}
\begin{itemize}
    \item Cormen, T. H., Leiserson, C. E., Rivest, R. L., \& Stein, C. (2009). \textit{Introduction to Algorithms} (3rd ed.). MIT Press.
    \item Sedgewick, R. (2011). \textit{Algorithms in C++}. Addison-Wesley.
    \item https://www.geeksforgeeks.org/randomized-algorithms-set-1-introduction-and-analysis/
    \item https://www.geeksforgeeks.org/randomized-algorithms-set-2-classification-and-applications/
    \item Lectures UIA
\end{itemize}

\end{document}
