\documentclass{article}
\usepackage{amsmath}
\usepackage{graphicx}
\usepackage{listings}
\usepackage{hyperref}
\usepackage{geometry}
\geometry{a4paper, margin=1in}
\usepackage{xcolor}

\lstset{
    language=C++,
    basicstyle=\ttfamily\small,
    keywordstyle=\color{blue},
    stringstyle=\color{red},
    commentstyle=\color{green},
    breaklines=true,
    frame=single,
    showstringspaces=false,
}

\title{Print Job Scheduling with Priority Queue Using Max-Heap}
\author{Gormery K. Wanjiru}
\date{\today}

\begin{document}

\maketitle

\section*{Print Job Scheduling System}

This program schedules print jobs by priority, using a max-heap to ensure higher-priority jobs are processed first. Users can add jobs, view the job with the highest priority, update priorities, and process jobs.


\section*{Components of the System}

1. \textbf{Priority Queue with Max-Heap}:
   - Jobs added to the max-heap are ordered by priority, so the job with the highest priority is at the top.
   - Includes methods for adding jobs, processing the highest-priority job, and updating job priorities.

2. \textbf{Adding Jobs}:
   - Users can add jobs with a name and priority, and the system shows the job with the highest priority after each addition.

3. \textbf{Updating Job Priority}:
   - Users can update any job’s priority, triggering a reorganization of the max-heap to maintain order.

   \section*{How to Use It}

   To use the print job scheduling program, follow these steps:
   
   \begin{enumerate}
       \item \textbf{Compile and Run}
       \begin{itemize}
           \item First, compile the program using a C++ compiler (e.g., \texttt{g++ main.cpp -o main}).
           \item Run the executable (e.g., \texttt{./main}) to start the program.
       \end{itemize}
   
       \item \textbf{User-Friendly Interface}
       \begin{itemize}
           \item Once the program is running, a menu interface will appear, guiding you through the options.
           \item The menu provides choices to add jobs, update priorities, process the highest-priority job, and view the job queue.
       \end{itemize}
   
       \item \textbf{Navigating the Menu}
       \begin{itemize}
           \item The program will prompt you to input your choice for each operation. Simply follow the prompts to:
           \begin{itemize}
               \item Add new print jobs with a name and priority.
               \item Update priorities for existing jobs.
               \item Process and remove the highest-priority job.
               \item View the full job queue and see which job has the highest priority.
           \end{itemize}
           \item After each operation, the program will display the current state of the queue, reflecting any changes made.
       \end{itemize}
   \end{enumerate}
   
   The interface ensures a smooth experience by handling input validation and displaying relevant error messages if needed.
   

\section*{Code Structure}

\subsection*{Main Priority Queue and Print Job Structure}

\begin{lstlisting}
// Priority Queue and Print Job Structure
#include <iostream>
#include <vector>
#include <string>
#include <unordered_map>

struct PrintJob {
    std::string name;
    int priority;
    PrintJob(const std::string& name, int priority) : name(name), priority(priority) {}
};

class PriorityQueue {
private:
    std::vector<PrintJob> heap;
    std::unordered_map<std::string, int> jobMap; // Maps names for fast access
    bool verbose;

    void heapifyUp(int index);
    void heapifyDown(int index);
public:
    PriorityQueue(bool verbose = true) : verbose(verbose) {}
    void insertJob(const std::string& name, int priority);
    void processJob();
    void updatePriority(const std::string& name, int newPriority);
    void displayHighestPriorityJob();
    void displayQueue();
};
\end{lstlisting}


\section*{Testing results}

Following the example scenario, we tested each feature in sequence. Below are the expected and actual outputs for each step, with the result noted as pass or fail.
\begin{enumerate}
    \item \textbf{Insert "Thesis" with Priority 4}
        \begin{itemize}
            \item \textbf{Expected:} \texttt{"Next job: Thesis (Priority: 4)"}
            \item \textbf{Actual:} \texttt{"Next job: Thesis (Priority: 4)"}
            \item \textbf{Result:} \textcolor{green}{PASS}
        \end{itemize}

    \item \textbf{Insert "Project" with Priority 5}
        \begin{itemize}
            \item \textbf{Expected:} \texttt{"Next job: Project (Priority: 5)"}
            \item \textbf{Actual:} \texttt{"Next job: Project (Priority: 5)"}
            \item \textbf{Result:} \textcolor{green}{PASS}
        \end{itemize}

    \item \textbf{Insert "Report" with Priority 3}
        \begin{itemize}
            \item \textbf{Expected:} \texttt{"Next job: Project (Priority: 5)"}
            \item \textbf{Actual:} \texttt{"Next job: Project (Priority: 5)"}
            \item \textbf{Result:} \textcolor{green}{PASS}
        \end{itemize}

    \item \textbf{Insert "Assignment" with Priority 2}
        \begin{itemize}
            \item \textbf{Expected:} \texttt{"Next job: Project (Priority: 5)"}
            \item \textbf{Actual:} \texttt{"Next job: Project (Priority: 5)"}
            \item \textbf{Result:} \textcolor{green}{PASS}
        \end{itemize}

    \item \textbf{Display Job with Highest Priority}
        \begin{itemize}
            \item \textbf{Expected:} \texttt{"Next job: Project (Priority: 5)"}
            \item \textbf{Actual:} \texttt{"Next job: Project (Priority: 5)"}
            \item \textbf{Result:} \textcolor{green}{PASS}
        \end{itemize}

    \item \textbf{Process Job with Highest Priority (Project)}
        \begin{itemize}
            \item \textbf{Expected:} \texttt{"Processing: Project (Priority: 5)"}
            \item \textbf{Actual:} \texttt{"Processing: Project (Priority: 5)"}
            \item \textbf{Result:} \textcolor{green}{PASS}
        \end{itemize}

    \item \textbf{Insert "Invoice" with Priority 6}
        \begin{itemize}
            \item \textbf{Expected:} \texttt{"Next job: Invoice (Priority: 6)"}
            \item \textbf{Actual:} \texttt{"Next job: Invoice (Priority: 6)"}
            \item \textbf{Result:} \textcolor{green}{PASS}
        \end{itemize}

    \item \textbf{Update Priority of "Thesis" to 7}
        \begin{itemize}
            \item \textbf{Expected:} \texttt{"Updated Thesis to priority 7"}
            \item \textbf{Actual:} \texttt{"Updated Thesis to priority 7. Queue contains: [Thesis (Priority: 7)] [Invoice (Priority: 6)] [Report (Priority: 3)] [Assignment (Priority: 2)]"}
            \item \textbf{Result:} \textcolor{green}{PASS}
        \end{itemize}
\end{enumerate}


\section*{Testing Code for Automated Sequence}

\begin{lstlisting}
    #define TESTING
    #include <iostream>
    #include <functional>
    #include <sstream>
    #include <string>
    #include "main.cpp" // Include the main.cpp file containing PriorityQueue
    
    // Helper function to capture output of specific actions
    std::string captureOutput(std::function<void()> func) {
        std::ostringstream buffer;
        std::streambuf* old = std::cout.rdbuf(buffer.rdbuf());
        func();
        std::cout.rdbuf(old);
        return buffer.str();
    }
    
    // Helper function to print test results
    void printTestResult(const std::string& testDescription, const std::string& actualOutput, const std::string& expectedOutput, bool condition) {
        std::cout << "Test: " << testDescription << "\n";
        std::cout << "Expected: " << expectedOutput << "\n";
        std::cout << "Actual: " << actualOutput << "\n";
        std::cout << (condition ? "Result: PASS" : "Result: FAIL") << "\n\n";
    }
    
    // Function to perform the example test sequence
    void runExampleTest() {
        PriorityQueue queue;
        bool condition;
        std::string output;
    
        // Step 1: Insert "Thesis" with Priority 4
        output = captureOutput([&]() { queue.insertJob("Thesis", 4); });
        condition = (output.find("Next job: Thesis (Priority: 4)") != std::string::npos);
        printTestResult("Insert 'Thesis' with priority 4", output, "Next job: Thesis (Priority: 4)", condition);
    
        // Step 2: Insert "Project" with Priority 5
        output = captureOutput([&]() { queue.insertJob("Project", 5); });
        condition = (output.find("Next job: Project (Priority: 5)") != std::string::npos);
        printTestResult("Insert 'Project' with priority 5", output, "Next job: Project (Priority: 5)", condition);
    
        // the other tests ...
    }
    
    int main() {
        runExampleTest();
        return 0;
    }
    
\end{lstlisting}

\section*{Testing Results discussion}
The testing sequence confirmed the system processes jobs by priority and manages priority updates correctly. Each function worked as expected.


\section*{Conclusion}

The priority queue works as intended, scheduling and processing print jobs based on priority. The max-heap ensures efficient management of jobs, and tests confirm that job insertion, priority updates, and processing operations behave as expected.

\section*{References}


\end{document}
